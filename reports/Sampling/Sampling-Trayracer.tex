\documentclass[a4paper,parskip=half,11pt]{scrartcl}

%%% Standardpakete
\usepackage[T1]{fontenc} 
\usepackage[utf8]{inputenc}
\usepackage[ngerman]{babel}
\usepackage{graphicx}

\usepackage{libertine} %Schriftart
\usepackage{tabularx} %fuer Tabellen
\usepackage{xcolor} %fuer Farben
\usepackage{amsmath} %fuer Mathematikmodus
%\usepackage[top=1cm]{geometry} %verringert Rand oben

\author{Trayracer: Oliver Kniejski, Steven Sobkowski, Marie Hennings}
\title{Bericht zum Sampling}

\begin{document}
 
\maketitle

\section*{Die Aufgabenstellung}

Die in diesem Bericht behandelte Aufgabenstellung befasst sich mit dem Thema Sampling.
Die Aufgabe ist die Implementierung eines Sampling-Patterns zum Zwecke des Anti-Aliasing.
Dazu soll zunächst eine zweidimensionale Point-Klasse angelegt werden, um sie in der Klasse SamplingPattern zu verwenden.
Diese SamplingPattern-Klasse wird von der Kamera genutzt. 

\section*{Lösungsstrategien}
Auch dieses Mal wurden alle Aufgaben gemeinsam bearbeitet und gelöst.

\section*{Implementierung}
Zunächst wurde die Point-Klasse implementiert und anschließend die SamplingPattern-Klasse.
Danach wurde die abstrakte Camera-Klasse erweitert und alle erbenden Kamera-Klassen angepasst.
Nachfolgend wurde die Raytracer-Klasse für die Entgegennahme mehrerer Rays modifiziert.
Als letztes mussten die vorhandenen Szenen angepasst werden.

\section*{Besondere Probleme oder Schwierigkeiten bei der Bearbeitung}
Keine :)

\section*{Zeitbedarf}
Die Aufgabe wurde an einem Nachmittag fertiggestellt.
Die dafür benötigte Stundenzahl beläuft sich etwa auf 4 Stunden pro Person.

\end{document}