\documentclass[a4paper,parskip=half,11pt]{scrartcl}

%%% Standardpakete
\usepackage[T1]{fontenc} 
\usepackage[utf8]{inputenc}
\usepackage[ngerman]{babel}

\usepackage{libertine} %Schriftart
\usepackage{tabularx} %fuer Tabellen
\usepackage{xcolor} %fuer Farben
\usepackage{amsmath} %fuer Mathematikmodus
%\usepackage[top=1cm]{geometry} %verringert Rand oben

\author{Trayracer: Oliver Kniejski, Steven Sobkowski, Marie Hennings}
\title{Bericht zu der Beleuchtung Teil 1}

\begin{document}
 
\maketitle

\section*{Die Aufgabenstellung}
In der aktuellen Aufgabe geht es um die Beleuchtung der Raytracerszene.
Zuerst sollen verschiedene Lichtquellen implementiert werden, die alle eine abstrakte Lichtklasse erweitern.
Die folgenden zu implementierenden Klassen sind:
\begin{itemize}
\item Punktlicht
\item Direktionales Licht
\item Spotlight
\end{itemize}

Des weiteren werden verschiedene Materialien implementiert, die wiederum die abstrakte Materialklasse erweitern:
\begin{itemize}
\item einfarbiges Material
\item Lambert Material
\item Phong Material
\end{itemize}

Infolgedessen müssen mehrere Klassen aus der vorherigen Aufgabe geändert oder ergänzt werden:
\begin{itemize}
\item Hit muss um eine Normale ergänzt werden
\item Geometry erhält statt dem Color-Attribut ein Material-Attribut
\item World wird um eine Liste für die Lichtguellen erweitert und bekommt ein weiteres Attribut für das ambiente Licht
\end{itemize}

Anschließend sollen zwei Szenen zur Überprüfung der neu implementierten Klassen erstellt werden.

\section*{Lösungsstrategien}

Vor Beginn der Bearbeitung der Aufgabe wurden alle Klassen auf die einzelnen Teammitglieder
aufgeteilt. Dabei wurde darauf geachtet, dass jeder etwa gleich komplexe Aufgaben erhält.
Anschließend wurden alle Klassen gemeinsam überprüft und mögliche Fehler und Probleme
zusammen gelöst.

\section*{Implementierung}

Die Klassen wurden anhand der in der Aufgabe vorgegebenen Klassendiagramme implementiert.
Die Hitmethoden der Geometrie-Klassen wurden um die Berechnung der Normalen erweitert.

\section*{Besondere Probleme oder Schwierigkeiten bei der Bearbeitung}
Keine :)

\section*{Zeitbedarf}
Der Zeitbedarf war bei dieser Aufgabe nicht sehr hoch. Insgesamt wurden 3-5 Stunden pro Person benötigt, um die Aufgabe fertigzustellen.

\end{document}