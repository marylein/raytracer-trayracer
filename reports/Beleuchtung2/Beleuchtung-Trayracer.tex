\documentclass[a4paper,parskip=half,11pt]{scrartcl}

%%% Standardpakete
\usepackage[T1]{fontenc} 
\usepackage[utf8]{inputenc}
\usepackage[ngerman]{babel}

\usepackage{libertine} %Schriftart
\usepackage{tabularx} %fuer Tabellen
\usepackage{xcolor} %fuer Farben
\usepackage{amsmath} %fuer Mathematikmodus
%\usepackage[top=1cm]{geometry} %verringert Rand oben

\author{Trayracer: Oliver Kniejski, Steven Sobkowski, Marie Hennings}
\title{Bericht zu der Beleuchtung Teil 2}

\begin{document}
 
\maketitle

\section*{Die Aufgabenstellung}
In der aktuellen Aufgabe geht es darum die Beleuchtung des Raytracers um Schattierung und Reflektion zu erweitern. 
Hierzu sollen die Konstruktoren der Lichtklassen um eine Variable vom Typ boolean ergänzt werden, welcher bestimmt, ob die Lichtquelle einen Schatten wirft.
Des Weiteren muss die \emph{Illuminates}-Methode der Lichtklassen angepasst werden.
Außerdem wird ein weiteres Material, das \emph{ReflectiveMaterial}, implementiert. 
Um die Tiefe der Rekursion bei reflektierenden Materialien zu berechnen, wird die Klasse \emph{Tracer} entworfen.

\section*{Lösungsstrategien}
Die Aufgaben wurden untereinander aufgeteilt und bei Problemen wurde versucht diese gemeinsam zu lösen.


\section*{Implementierung}
Die Aufgaben wurden anhand der Aufgabenstellung implementiert. Lediglich der Tracer hatte spezifischen Vorgaben. Dieser erhielt das Attribut \emph{Counter} welcher im Konstruktor übergeben wird und die Rekursionstiefe bestimmt. Des Weiteren gibt es die Methode Tracing, welche ein Color-Objekt zurückgibt und nur vom \emph{ReflectiveMaterial} aufgerufen wird.

\section*{Besondere Probleme oder Schwierigkeiten bei der Bearbeitung}
Vorübergehende Schwierigkeiten hatten wir bei der Implementierung des \emph{Tracers}, im Speziellen bei dem rekursiven Aufruf der Reflektionen.

\section*{Zeitbedarf}
Der Zeitbedarf betrug pro Person ca. 7 Stunden.

\end{document}